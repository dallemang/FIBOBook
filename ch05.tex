
%blankpage

\chapter{Math}
\label{ch05}

\section{Bold Math}
Sometimes you have a math symbol in bold to indicate, for example,
a vector or special functions. Use \verb+\mathbf{x}+ for letters
and \verb+\bm{\Phi}+ for Greek letters.

\noindent
\ldots
$\bm{\Phi}$ and $\bm{\Theta}$ are the distribution functions
governing the likelihood of $\mathcal{X}$ and $\mathbf{Z}$ \ldots

\blankline
Some notes:
\begin{itemize}
\item
use \verb+\bm{\Phi}+ to produce a bold Greek symbol Phi.
\item
use \verb+\mathcal{X}+ to display the caligraphic letter X
\item
use \verb+\mathbf{Z}+ to dispaly bold non-Greek symbols
\end{itemize}

\blankline
If you plan to use Greek symbols to identify, for example, vectors, you could
create a macro, like

\verb+\newcommand{\vect}[1]{\bm{#1}}+

\newcommand{\vect}[1]{\bm{#1}}

and then use

\noindent
\begin{verbatim}
\ldots the vectors are $\vect{\Phi}$ and $\vect{\Theta}$ \ldots
\end{verbatim}

to produce

\noindent
\ldots the vectors are $\vect{\Phi}$ and $\vect{\Theta}$ \ldots

\blankline
The idea is to have macro names for different constructs.
If you chose to use a different representation for vectors,
you change the macro and rerun your manuscript.


\section{Math Variables}

When we have a math variable with more than one character,
like \textit{abcd}, \LaTeX{} treats each character as a separate variable.
With Caslon fonts, that we use to produce your book, that space is noticeable.

We eliminat the extra space using \verb+\mathit+:

\verb+\mathit{abcd} = 3.+
$$
\mathit{abcd} = 3.
$$

If we use the variable \textit{abcd} often, as in

\verb|\mathit{abcd} = \mathit{abcd}^2 - \sqrt{\mathit{abcd}} + \mathit{abcd},|

$$
\mathit{abcd} = \mathit{abcd}^2 - \sqrt{\mathit{abcd}} + \mathit{abcd},
$$
we can create a separate macro for the variable:

\verb+\newcommand{\abcd}{\mathit{abcd}}+
\newcommand{\abcd}{\mathit{abcd}}

\noindent
Then we can use \verb+\abcd+ in the expressions:

\verb|\abcd = \abcd^2 - \sqrt{\abcd} + \abcd,|

$$
\abcd = \abcd^2 - \sqrt{\abcd} + \abcd,
$$

If there is already a predefined command with the same name,
we use instead something like \verb+Abcd+:

\verb+\newcommand{\Abcd}{\mathit{abcd}}+
\newcommand{\Abcd}{\mathit{abcd}}

\noindent
Then we can use \verb+\Abcd+ in the expressions:

\verb|\Abcd = \Abcd^2 - \sqrt{\Abcd} + \Abcd.|

$$
\Abcd = \Abcd^2 - \sqrt{\Abcd} + \Abcd.
$$


\clearpage

