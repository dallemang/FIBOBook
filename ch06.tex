
%blankpage

\chapter{Citations with Author-Year}
\label{ch06}

An alternative to numerical \LaTeX\ citations makes reference to published
works by citing the author's name and year of publication. The entries in 
the bibliographic listing are not numbered. The citation itself may be 
either parenthetical like \citep{carroll98} or textual as shown by
\citet{carroll98}.

The \verb+natbib+ package by Patrick W. Daly is the most universal package
for Author-Year citation and it is compatible with \verb+\bibitem+ syntaxes
and with the \verb+.bst+ files of author-year packages such as
\verb+apalike+, \verb+chicago+, and \verb+harvard+.

If you create a bibliography by hand, please see the file \verb+biblio.tex+
in the \verb+latex+ directory of \verb+bib-AY.zip+.
The \verb+biblio.tex+ files shows the syntax that \verb+natbib.sty+ 
requires and notice how we use
that package in~\verb+book.tex+ and in~\verb+ch01.tex+. 
You can find the details about \verb+natbib+ and Author-Year bibliographies
in~\citet[pages~218-221]{KD:2004}.

Note that we have bibliographic references like \verb+carroll98+,
\verb+KD:2004+, and \verb+weber-97+. Pick a style that works for you
and stay with it, then it will be easier for yourself and others to 
maintain and modify your bibliography and citations.

If you use \verb+.bib+ files then you should know about \verb+BibTeX+,
\verb+.bst+ files, and you should know how to create the \verb+.bbl+
files as well.
(If you use \verb+.bib+ files but you do not know the rest, you should
consult a local expert before continuing with \verb+.bib+ or create
your bibliography by hand.)
If you know the steps we mentioned above and you wish to create an 
Author-Year bibliography and citations, then you are in luck: 
there are three \verb+.bst+ that are provided with \verb+natbib+.
They are: \verb+plainnat+, \verb+unsrtnat+, and \verb+abbrvnat+.

Once you create your \verb+.bbl+ file(s), please submit them along with
your manuscript files, style files, and (preferably) \verb+.eps+ figures with 
\verb+BoundingBox+es to \textit{Morgan and Claypool Publishers.}

You can get a free copy of the \verb+natbib+ package  from CTAN at
\begin{center}
\verb+http://www.ctan.org/tex-archive/macros/latex/contrib/natbib/+
\end{center}
and download the file \verb+natbib.zip+.

Here are some more samples using the citations that appear in the file
\verb+biblio.tex+ that we built by hand.

The next paragraph uses text citations using \verb+\citet+ (cite as text):\\
\noindent
There are several textbooks that give a general 
introduction to dependency grammar but
most of them in other languages than English, for example,
\citet{tarvainen82} and~\citet{weber-97}.

The last paragraph uses text citations with parentheses
using \verb+\citetp+:\\
\noindent
Tesni{\`e}re's seminal work was published
posthumously as~\citep{tesniere59}. 
Other influential theories in the dependency
grammar tradition include Functional Generative
Description~\citep{SHP:1986}.

Please see the bibliography on the next page and the file \verb+biblio.tex+.

\clearpage

